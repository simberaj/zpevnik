\beginsong{O malém rytíři}[by={Traband}]
\chor
\[Hm]Jede jede rytíř, \[D]jede do kraje
\[A]Nové dobrodružství \[Hm]{v dálce} hledaje
\[Hm]Neví, co je bázeň, \[D]neví, co je strach
\[A]Má jen velké srdce \[Hm]{a na botách} prach
\cl\num
\[G]Jednou takhle v neděli, \[F\shrp{}]slunce pěkně hřálo
\[G]Bylo kolem poledne, \[F\shrp{}]když tu se to stalo
\[G]Panáček uhodí \[A]pěstičkou do stolu:
\[G]Dosti bylo pohodlí \[A]{a plnejch} kastrólů!
\[Hm]Ještě dneska stůj co stůj \[A]musím na cestu se dát
\[G]Tak zavolejte sloužící a \[F\shrp{}\hidx{7}]dejte koně osedlat!
\fin\chordsoff\num
Ale milostpane! spráskne ruce starý čeledín
Ale pán už sedí v sedle a volá s nadšením:
Má povinnost mi velí pomáhat potřebným
Ochraňovat chudé, slabé, léčit rány nemocným
Marně za ním volá stará hospodyně:
Vraťte se pane, lidi jsou svině!
\fin\num
Ale sotva dojel kousek za městskou bránu
Z lesa na něj vyskočila banda trhanů
Všichni ti chudí, slabí, potřební~-- ta chátra špinavá
Vrhli se na něj a bili ho hlava nehlava
Než se stačil vzpamatovat, bylo málem po něm
Ukradli mu co kde měl a sežrali mu koně
\fin\num
Vzhůru srdce! zvolá rytíř, nekončí má pouť
Svou čest a slávu dobudu, jen z cesty neuhnout!
Hle, můj meč! (a zvedl ze země kus drátu)
A zde můj štít a přilbice! (plechovka od špenátu)
Pak osedlal si pavouka, sed na něj, řekl hyjé
Jedem vysvobodit princeznu z letargie!
\fin\num
A smutná princezna, sotva ho viděla
Vyprskla smíchy a plácla se do čela
Začala se chechtat, až jí z očí tekly slzy
To je neskutečný, volala, jak jsou dneska lidi drzý!
O mou ruku se uchází tahle figura
Hej, zbrojnoši, ukažte mu rychle cestu ze dvora!
\fin\num
Tak jede malý rytíř svojí cestou dál
Hlavu hrdě vzhůru~-- vždyť on svou bitvu neprohrál
I když král ho nechal vypráskat a drak mu sežral boty
A děvka z ulice mu plivla na kalhoty
Ve světě, kde lidi na lidi jsou jak vlci
Zůstává rytířem~-- ve svém srdci
\fin
\endsong



