\beginsong{Darling}[by={Vypsaná fixa}]
\emptyv
\cseq{\[D] \[A] \[Em] \[G]}\\
\cl
\chordsoff
\num
V bodu, kdy ti zmizí úplně všechno si já vymyslím nejsladší pointy
Někdo to neřeší, někdo pije drinky, někdo v tom bodu vykouří jointy
Já mám rád hrdiny, kteří lezou po dně a v poslední chvíli se zvednou
Někdy taky dělám, že jsem vyřízenej a je mi to úplně jedno
\fin
\bridge
\chordson
\[G\hidx{7}]{Tak}\[D\hidx{7}]{ js}\[A\hidx{7}]em \[D5]řek
\chordsoff
\reppart{Dno je dobrý na to, aby sis ho prohlíd}
\cl
\chor
Tak jsem řek~-- hej, máš drobný, darling a ona řekla že nějaký má
A já jsem řek~-- fajn, tak na dvě kávy z automatu, on nám je dá
A pak jsem šli ven a tam jsme stáli tři hodiny a dvacet minut
A kolem byl svět, který byl správný i s kelímkama z automatu
\cl
\num
V bodu, kdy ti zmizí úplně všechno jsem byl, tam to znám velmi
V podstatě to tam má svojí poetiku a originální genius loci
Lítají tam pořad ty samý bumerangy, který se pokaždý vrátěj
Uprostřed toho všeho mezi Strážnou věží a pornografickým stánkem
\fin
\bridge\emptyspace\\ \cl
\chor
Tak jsem řek~-- hej, smím prosit, darling a ona řekla, že jsem to já
A já sem řek~-- fajn, tak je to správný, doktor Jekyll zvítězí rád
A pak jsem řek~-- hej, chtěl bych tě, darling, a ona řekla~-- co bude dál
A hráli jsme squash a taky carving, uklízet a zaparkovat
\cl
\cverse
\reppart{A tak jsem řek~-- hej, to bude dál
A pak jsem řek~-- hej, ale je tu i Hyde}
\cl
\endsong


