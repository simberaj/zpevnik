\beginsong{Okoř}
\num
\[D]{Na Okoř} je cesta jako žádná ze sta
\[A\hidx{7}]vroubená je stroma\[D]{ma \emph{(stromama)}}
\[D]Když jdu po ní v létě, samoten na světě
\[A\hidx{7}]sotva pletu noha\[D]{ma \emph{(nohama)}}
\[G]{Na konci} té cesty \[D]trnité \[E]stojí krčma jako \[A\hidx{7}]hrad \emph{(jako hrad)}
\[D]Tam zapadli trampi, hladoví a sešlí, \[A\hidx{7}]začli sobě notov\[D]at
\fin
\chordsoff
\chor
\chordson
\[D]{Na hradě} Okoři \[A\hidx{7}]světla už nehoří
\[D]Bílá paní \[A\hidx{7}]{šla už} dávno s\[D]pát
Ta měla ve zvyku \[A\hidx{7}]podle svého budíku
\[D]{o půlnoci} \[A\hidx{7}]chodit straší\[D]vat
\[G]{Od těch} dob, co jsou tam \[D]trampové
\[E]nesmí z hradu \[A\hidx{7}]pryč \emph{(z hradu pryč)}
\[D]{A tak} dole v podhradí \[A\hidx{7}]{se šerifem} dovádí
\[D]{on jí} sebral \[A\hidx{7}]{od komnaty} \[D]klíč
\cl
\num
Jednoho dne zrána roznesla se zpráva, že byl Okoř vykraden
Nikdo neví dodnes, kdo to tenkrát odnes, nikdo nebyl dopaden
Šerif hrál celou noc mariáš s bílou paní v kostnici
Místo aby hlídal, zuřivě jí líbal, dostal z toho zimnici
\fin
\repchorus{\emptyspace}
\endsong


