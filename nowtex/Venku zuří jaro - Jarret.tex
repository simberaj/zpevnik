\beginsong{Venku zuří jaro}[by={Jarret}]
\num
\[E]Sleduju, jak jeden pán u okýnka usíná
\[F\shrp{}m]Klimbá a slintá, něco se mu \[E]zdá
S trhnutím pak procitá a jak \[F\shrp{}m]tráva deštěm pobitá
Vstává, \[G\shrp{}m]{před ústa} si ruku \[A]dává, někam \[H]utí\[E]ká
\fin\chordsoff\num
Na jeho místo k oknu usedá žena, kterou dobře znám
Z obálky Květů, jež držel tamten pán
Do skla na sebe se podívá a hlavou smutně pokývá
Ví, že už jen musí, že žádný \emph{můžeš} neplatí
\fin\chor
\chordson
A ona \[H]ví\[C\shrp{}m]í, že \[A]nehřeje, že nechla\[E]dí
A \[A]na dotek je, jak se \[H]zdá, tak ako\[E]rát
Venku zuří jaro, \[A]já mám jednu starost, kdy \[E]víceletku zasadím
Chce to nějakou změnu, \[A]to, co nedoženu
\[E]{Z dohledu} snad neztratím
\cl\num
Kam zmizel onen spící muž, co ho bodá jako nůž
Nůž ostrý jako kletba, jako urážka
Krátký sen uprostřed dne ho nechá, ať si vzpomene
Na to, na co myslet nechce, ale spíše ne
\fin\num
A ta žena možná ví, že ví, stejně z toho nesleví
Už je přeci pozdě už nic nezmění
Zpívám si spolu s ní o věcech prvních posledních
Místo tečky bože chtěl bych radši vykřičník
\fin\chor
I když vím, že nehřeju, že nechladím
A na dotek jsem, jak se zdá, tak akorát
Týdyjada\ldots
\cl
\endsong



