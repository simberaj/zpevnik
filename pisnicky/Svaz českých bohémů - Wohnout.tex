\beginsong{Svaz českých bohémů}[by={Wohnout}]
\emptyv
\cseq{\[G] \[Dm] \[Am] \[F] \[C]}\\
\cl
\chordsoff
\chor
\chordson
\[G]Vracím se domů nad ránem \[Dm]kvalitním vínem omá\[Am]men
Z přímek se stávaj zatá\[F]čky, točí se \[C]svět, jsem na sra\[G]čky
\chordsoff
Vedle mě zvrací princezna, nastavím dlaň a požehnám
Pro všechny jasný poselství~-- tomu se říká přátelství
\cl
\num
Au~-- mám tisíc otázek a žádnou vzpomínku
Skládám si obrázek kámen po kamínku
Včerejší produkce šla do bezvědomí
Nastává dedukce, co na to svědomí
A už si vzpomínám, já byl jsem na srazu
S kumpány který mám, patříme do svazu
Vlastníme doménu, tak si nás rozklikni
Svaz českejch bohémů
\fin
\repchorusi{\emptyspace}
\num
Stačí pár večírků společně s bohémy
za kterými se táhnou od školy problémy
V partě je Blekota, Jekota, Mekota
dost často hovoříme o smyslu života
Jako je třeba teď, mám tisíc otázek
A žádnou vzpomínku, si skládám obrázek
Z těžkejch ran lížu se, včera jsme slavili
svatýho Vyšuse
\fin
\repchorusi{\emptyspace}
\num
Svět zrychluje svý otáčky, sousedka peče koláčky
Hlášen stav nouze nejvyšší, Hapkové volaj Horáčky
Zástupy českejch bohémů vyráží ven do terénu
Šavlí z kvalitního vína bojovat proti systému
\fin
\chorusii
\reppart{Tak jsme se tu všichni sešli, co myslíš, osobo
Na to nelze říci než~-- co je ti do toho
Tak jsme se tu všichni sešli, co myslíš, osude
Na to nelze říci než~-- jinak to nebude} \rep{3}
\cl
\endsong


