\beginsong{Těšínská}[by={Jaromír Nohavica}]
\num
\[Am]Kdybych se narodil \[Dm]před sto léty \[F]v \[E]tomhle \[Am]městě \[Dm] \[F] \[E] \[Am]
U Larischů na zahradě \[Dm]trhal bych květy \[F]
\[E]své ne\[Am]věstě \[Dm] \[F] \[E] \[Am]
\[C]Moje nevěsta by \[Dm]byla dcera ševcova
\[F]Z domu Kamińskich \[C]odněkud ze Lvova
Kochal bym ja i \[Dm]pieščil, \[F]chy\[E]ba lat \[Am]dwiešči \[Dm] \[F] \[E] \[Am]
\fin\chordsoff\num
Bydleli bychom na Sachsenbergu v domě u žida Kohna
Nejhezčí ze všech těšínských šperků byla by ona
Mluvila by polsky a trochu česky
Pár slov německy a smála by se hezky
Jednou za sto let zázrak se koná, zázrak se koná
\fin\num
Kdybych se narodil před sto lety, byl bych vazačem knih
U Prohazků dělal bych od pěti do pěti
a sedm zlatek za to bral bych
Měl bych krásnou ženu a tři děti
Zdraví bych měl a bylo by mi kolem třiceti
Celý dlouhý život před sebou, celé krásné dvacáté století
\fin\num
Kdybych se narodil před sto lety v jinačí době
U Larischů na zahradě trhal bych květy, má lásko, tobě
Tramvaj by jezdila přes řeku nahoru
Slunce by zvedalo hraniční závoru
A z oken voněl by sváteční oběd
\fin\num
Večer by zněla od Mojzese melodie dávnověká
Bylo by léto tisíc devět set deset, za domem by tekla řeka
Vidím to jako dnes~-- šťastného sebe
Ženu a děti a těšínské nebe
Ještě že člověk nikdy neví, co ho čeká
\chordson
\[Am]{Na na na} \[Dm]{na na na} \[F]na na \[E]na na \[Am]na
\chordsoff
\fin
\endsong



