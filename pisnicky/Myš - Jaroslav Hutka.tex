\beginsong{Myš}[by={Jaroslav Hutka}]
\num
V \[C]kuchyni se \[G]řízek \[C]smažil,
\[F]kuchař okolo něj \[C]tan\[G]čil,
v \[C]něžném ryt\[F]{mu valčí}\[G]ku
\[C]rozpalo\[G]val pánvič\[C]ku, \[G]  \[C]rozpalo\[G]val pánvič\[C]ku.
\fin
\chordsoff
\num
Najednou omastek prsknul, kuchař sebou na zem mrsknul
a tím, jak se dostal níž, [: spatřil dole šedou myš :]
\fin
\num
Pískala si na hoboj~-- \uv{Na krásnom modrom Dunaji},
dupala si nožičkou [: a škubala kůžičkou :]
\fin
\num
Kuchař sáhnul po lopatě, usmál se nasládle, svatě,
a než se myš nadála, [: v omastku kraulovala :]
\fin
\num
Že pevný charakter měla, na hoboj pořád hrála,
už byla prosmažená, [: ale ne poražená :]
\fin
\num
Už se na talířku nese, pod vrchním se země třese,
strávník si ji prohlídnul [: a vidličku zabodnul :]
\fin
\num
Vznes' ji k ústům labužnicky, zakousl se nostalgicky,
polknul nahlas jako ras, [: vtom uslyšel zevnitř hlas :]
\fin
\num
Ty balvane nekulturní, ještě si nakonec krkni,
moji píseň nezničíš, [: jenom počkej, uvidíš! :]
\fin
\num
V žaludku se cosi hnulo, strávníka to vylekalo,
střeva místo trávení [: dala se do zpívání :]
\fin
\num
Začaly zpívat kosti, cévy, co si počnout, chlapík neví,
tu, ke svému zděšení, [: pustil se do tančení :]
\fin
\num
Roztančil celičký lokál, zmateně odtamtud prchal,
kudy běžel, tudy hrál, [: nikdo mu neodolal :]
\fin
\num
A když dotančili k řece, do vody skákali křepce,
myši zas šťastně žily, že se lidí zbavily,
a město vyčistily, docela šťastný byly,
že se lidí zbavily, a město vyčistily.
\fin
\endsong




