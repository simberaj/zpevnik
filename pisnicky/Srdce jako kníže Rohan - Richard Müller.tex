\beginsong{Srdce jako kníže Rohan}[by={Richard Müller}]
\num
\[F]Měsíc je jak Zlatá bula \[C]sicilská
\[Am]Stvrzuje, že kdo chce, ten se \[G]dopíská
\[F]Pod lampou jen krátce, v přítmí \[C]dlouze zas
\[Am]Otevře ti Kobera a \[G]můžeš mezi \[F]nás
\fin
\chordsoff
\num
Moje teta, tvoje teta, parole
Dvaatřicet karet křepčí na stole
Měsíc svítí sám a chleba nežere
Ty to ale koukej trefit, frajere \emph{(protože)}
\fin
\chorusi
Dnes je valcha u starýho Růžičky
Dej si prachy do pořádný ruličky
Co je na tom že to není extra nóbl byt
Srdce jako kníže Rohan musíš mít
\cl
\num
Ať jsi přes den docent nebo tunelář
Herold svatý pravdy nebo jinej lhář
Tady na to každej kašle zvysoka
Pravda je jen jedna \emph{(slova proroka říkaj, že)}
\fin
\chor
Když je valcha u starýho Růžičky
Budou vcelku nanic všechny řečičky
Buď to trefa nebo kufr~-- smůla nebo šnit
Jen to srdce jako Rohan musíš mít
\cl
\num
Kdo se bojí má jen hnědý kaliko
Možná občas nebudeš mít na mlíko
Jistě ale poznáš co jsi vlastně zač
Svět nepatřil nikomu, kdo nebyl hráč \emph{(a proto)}
\fin
\chorusii
Ať je valcha u starýho Růžičky
nebo pouť až k tváři Boží rodičky
Ať je válka, červen, mlha, bouřka nebo klid
Srdce jako kníže Rohan musíš mít
\cl
\chor
Dnes je valcha u starýho Růžičky
když jsi malej, tak si stoupni na špičky
Malej nebo nachlapenej, cikán, prdák, žid
Srdce jako kníže Rohan musíš mít
\cl
\chorusi
\rep{2}
\cl
\endsong


