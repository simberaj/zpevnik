\beginsong{Akordy}[by={Karel Plíhal}]
\begin{textblock}{5}(9,-0.5) \gtab{A\hidx{maj7}}{002120:002130} \end{textblock}
\num
\[E]Nejkrásnější akord bude \[A\hidx{maj7}]A-maj,
\[E]prstíky se při něm nepo\[C]lámaj.
\[A]Pomohl mi k \[H\hidx{7}]pěkné holce s \[G\shrp{}m\hidx{7}]absolutním \[C\shrp{}m]sluchem,
\[A]každý večer \[C]naplníme \[E]balón horkým \[A\hidx{maj7}]vzduchem.
\fin\chordsoff\num
V stratosféře hrajeme si A-maj,
i když se nám naši známí chlámaj.
Potom, když jsme samým štěstím opilí až namol,
\chordson
stačí místo A-maj jenom zahrát třeba \[Am]A-moll.
\fin\num
\chordson
A-moll všechny city rázem \[Am]zchladí,
\chordsoff
dopadneme na zem na pozadí.
Sedneme si do trávy a budem koukat vzhůru,
dokud nás čas nenaladí aspoň do A-duru.
\fin\chor\chordson
\[A]Od A-dur je jenom kousek k \[A\hidx{maj7}]A-maj,
\[F\shrp{}m]proto všem těm, \[F\shrp{}m/F]{co se v lásce} \[F\shrp{}m/E]zklamaj:\[F\shrp{}m/D]
\[A]vyždímejte \[H\hidx{7}]kapesníky a \[G\shrp{}m]nebuďte \[C\shrp{}m]smutní,
\[A]každá holka \[C]pro někoho \[E]má sluch abso\[A\hidx{maj7}]lutní.
\chordsoff
\cl\num
V každém akord zní, aniž to tuší,
zkusme tedy nebýt k sobě hluší.
Celej svět je jeden velkej koncert lidských duší,
jenže jako A-maj nic tak srdce nerozbuší.
\fin\num
\chordson
\[E]Pro ty, co to A-maj v lásce \[A\hidx{maj7}]nemaj',
\[E]moh bych zkusit zahrát třeba \[C\hidx{maj7}]C-maj.
\chordsoff
\fin
\endsong




